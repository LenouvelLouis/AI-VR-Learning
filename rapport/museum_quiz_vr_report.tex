\documentclass[11pt,a4paper]{report}
\usepackage[utf8]{inputenc}
\usepackage[T1]{fontenc}
\usepackage{graphicx}
\usepackage[margin=2.2cm]{geometry}
\usepackage{hyperref}
\usepackage{xcolor}
\usepackage{float}
\usepackage{enumitem}
\usepackage{fancyhdr}
\usepackage{titlesec}
\usepackage{booktabs}
\usepackage{tabularx}
\usepackage{parskip}
\usepackage{caption}

% Compact chapter titles
\titleformat{\chapter}[hang]{\normalfont\LARGE\bfseries}{\thechapter.}{1em}{}
\titlespacing*{\chapter}{0pt}{-10pt}{15pt}
\titleformat{\section}{\normalfont\large\bfseries}{\thesection}{1em}{}
\titlespacing*{\section}{0pt}{10pt}{5pt}
\titleformat{\subsection}{\normalfont\normalsize\bfseries}{\thesubsection}{1em}{}
\titlespacing*{\subsection}{0pt}{8pt}{3pt}

% Colors
\definecolor{codeblue}{RGB}{0,102,204}
\hypersetup{colorlinks=true,linkcolor=codeblue,urlcolor=codeblue}

% Header/Footer
\pagestyle{fancy}
\fancyhf{}
\fancyhead[L]{\small\leftmark}
\fancyhead[R]{\small Museum Quiz VR}
\fancyfoot[C]{\thepage}
\renewcommand{\headrulewidth}{0.4pt}

% Compact lists
\setlist{noitemsep,topsep=2pt}

\begin{document}

% Title Page
\begin{titlepage}
\centering
\vspace*{1.5cm}
{\scshape\Large XR Development Project\\Unity + VR + AI\par}
\vspace{1.2cm}
{\Huge\bfseries Museum Quiz VR\par}
\vspace{0.4cm}
{\large\itshape An Educational VR Application for Meta Quest 3S\par}
\vspace{1.5cm}
{\large\textbf{Team Members:}\par}
\vspace{0.3cm}
{\normalsize
Louis Lenouvel $\cdot$ Louis Grignola $\cdot$ Gabriel Esteves\\[3pt]
Gabriel Benhamou $\cdot$ Raphael Lifergan $\cdot$ Lisa $\cdot$ Nina\par}
\vspace{1.5cm}
{\large January 2026\par}
\vfill
\end{titlepage}

\tableofcontents
\newpage

% Executive Summary
\chapter*{Executive Summary}
\addcontentsline{toc}{chapter}{Executive Summary}

This report documents the seven-week development of \textbf{Museum Quiz VR}, an educational virtual reality application for Meta Quest 3S. The project combined VR technology with AI-powered quiz generation to create an engaging learning experience about world monuments.

\textbf{Key outcomes:}
\begin{itemize}
    \item A functional VR museum with 31 interactive monument paintings
    \item Dynamic quiz generation using Google Gemini API
    \item Ergonomic VR user interface with wrist-mounted HUD
    \item Complete game loop with scoring, timer, and win conditions
\end{itemize}

The project underwent a significant pivot in Week 4, transitioning from an ambitious real-time object detection concept to a more achievable museum quiz experience. This decision proved crucial for delivering a polished final product within the constraints of limited VR hardware access (single headset for a 7-person team).

\chapter{Week 1 -- Project Vision \& Ideation}

\section{Group Session Work}

The first session focused on establishing the project vision. With a team composition featuring three AI specialists, we naturally gravitated toward AI-centric use cases.

\subsection{Statement of Purpose}

After brainstorming various concepts, we defined our mission:

\begin{quote}
\textit{``The mission of this project is to design an immersive educational application for the Meta Quest 3S virtual reality headset that leverages Virtual Reality (VR) and Artificial Intelligence (AI) to enhance cultural learning. The application places users inside a virtual museum where they can explore historical monuments and artworks. By interacting with these elements, users are challenged through an AI-generated quiz that evaluates their knowledge and provides immediate feedback.''}
\end{quote}

\subsection{Success Criteria}

We established measurable success criteria:
\begin{itemize}
    \item Users can intuitively explore the virtual museum
    \item Users actively engage with AI-generated quizzes
    \item Users learn cultural and historical information through immersive interaction and clear feedback
\end{itemize}

% Figure: Purpose and Success Criteria
\begin{figure}[H]
\centering
\includegraphics[width=0.85\textwidth]{images/purpose_success.png}
\caption{Statement of Purpose and Success Criteria defined in Week 1}
\label{fig:purpose}
\end{figure}

\section{Individual/Subgroup Work}

\subsection{AI Proof of Concept}

The AI subgroup (Louis Lenouvel, Louis Grignola, Gabriel Esteves) developed a proof-of-concept on PC:
\begin{itemize}
    \item Integrated Google Vision API for object detection
    \item Tested prompts with Google Gemini for contextual responses
    \item Validated the pipeline: image capture $\rightarrow$ detection $\rightarrow$ text generation
\end{itemize}

The POC successfully detected monuments from images and generated relevant information, demonstrating technical feasibility.

% Figure: Early AI Test
\begin{figure}[H]
\centering
\includegraphics[width=0.7\textwidth]{images/early_ai_test.png}
\caption{Initial AI POC: Eiffel Tower detection with Gemini generating contextual information}
\label{fig:early_ai}
\end{figure}

\section{Problems \& Solutions}

\begin{tabularx}{\textwidth}{|l|X|X|}
\hline
\textbf{Problem} & \textbf{Impact} & \textbf{Solution} \\
\hline
LLM selection & Uncertainty about costs and integration & Selected Gemini for ease of use and pricing \\
\hline
API latency & Slow responses affecting UX & Implemented loading indicators \\
\hline
\end{tabularx}

\section{Week Outcome}

\textbf{Achieved:} POC validated, architecture defined, roles assigned\\
\textbf{Decision:} Proceed with Quest 3S integration

% Figure: Kanban Week 1
\begin{figure}[H]
\centering
\includegraphics[width=0.95\textwidth]{images/kanban_week1.png}
\caption{Kanban board at end of Week 1 showing initial task distribution}
\label{fig:kanban1}
\end{figure}

\chapter{Week 2 -- POC Validation \& VR Constraints}

\section{Group Session Work}

The session began with a POC demonstration to the full team. Results were deemed convincing, validating our technical approach.

Key discussions covered:
\begin{itemize}
    \item Integration strategy for Meta Quest 3S
    \item Camera passthrough access requirements
    \item Detailed role assignments
\end{itemize}

\subsection{Team Roles Definition}

\begin{table}[H]
\centering
\begin{tabularx}{\textwidth}{|l|l|X|}
\hline
\textbf{Role} & \textbf{Members} & \textbf{Responsibilities} \\
\hline
AI \& API Development & Louis L., Louis G., Gabriel E. & Gemini integration, quiz generation, response validation \\
\hline
Unity Integration & Louis Lenouvel & Scene setup, VR interaction, GitHub coordination \\
\hline
Gameplay \& Game Loop & Gabriel B., Raphael L. & Rules, objectives, user flow, diagrams \\
\hline
UI/UX Design & Lisa, Nina & VR interface design, ergonomics, visual feedback \\
\hline
\end{tabularx}
\caption{Team roles and responsibilities}
\end{table}

% Figure: Roles
\begin{figure}[H]
\centering
\includegraphics[width=0.9\textwidth]{images/roles.png}
\caption{Roles \& People diagram showing team organization}
\label{fig:roles}
\end{figure}

\section{Individual/Subgroup Work}

Louis Lenouvel began Quest 3S integration attempts:
\begin{itemize}
    \item Meta XR SDK installation in Unity
    \item Project configuration for Quest builds
    \item Initial passthrough access tests
\end{itemize}

Other team members researched Meta Quest camera APIs and documented existing solutions.

\section{Problems \& Solutions}

\begin{tabularx}{\textwidth}{|l|X|X|}
\hline
\textbf{Problem} & \textbf{Impact} & \textbf{Solution} \\
\hline
Complex Meta SDK & Steep learning curve & Focus on official examples \\
\hline
Single headset & Testing bottleneck & Prioritize headset time for critical tests \\
\hline
\end{tabularx}

\section{Week Outcome}

\textbf{Achieved:} POC presented, team aligned, roles assigned\\
\textbf{Decision:} Continue integration despite emerging difficulties

\chapter{Week 3 -- Integration Blockers}

\section{Group Session Work}

This week was marked by significant technical challenges. The session focused on collective debugging and problem analysis.

\subsection{Identified Issues}

\begin{enumerate}
    \item \textbf{Passthrough camera access:} Meta's API doesn't easily allow frame extraction for external processing
    \item \textbf{Permission complexity:} Undocumented configuration requirements
    \item \textbf{Performance concerns:} Initial tests showed significant latency
\end{enumerate}

\section{Individual/Subgroup Work}

Louis Lenouvel tested multiple approaches:
\begin{itemize}
    \item Different SDK versions
    \item Third-party camera access plugins
    \item Meta developer forum solutions
\end{itemize}

The team systematically documented all errors, configurations tested, and abandoned approaches.

\section{Problems \& Solutions}

\begin{tabularx}{\textwidth}{|l|X|X|}
\hline
\textbf{Problem} & \textbf{Impact} & \textbf{Solution} \\
\hline
Hardware dependency & Cannot parallelize VR testing & Document issues for async work \\
\hline
SDK bugs & Unpredictable passthrough behavior & Test alternative approaches \\
\hline
Time loss & Multiple hours with no progress & Prepare contingency plan \\
\hline
\end{tabularx}

\subsection{Risk Assessment}

The team identified a critical risk: the initial concept might not be achievable within the timeframe given hardware constraints. We decided to prepare alternative approaches.

\section{Week Outcome}

\textbf{Not achieved:} Passthrough camera integration\\
\textbf{Achieved:} Complete documentation of blockers, risk awareness\\
\textbf{Decision:} If Week 4 doesn't unblock, consider pivot

\chapter{Week 4 -- Strategic Pivot}

\section{Group Session Work}

Facing persistent blockers, we explored an alternative: \textbf{Street View-style immersive detection} using geolocated panoramic images instead of real-time camera.

\subsection{Street View Concept Testing}

\begin{itemize}
    \item Integrated 360° images in Unity
    \item Attempted element detection in panoramas
    \item Analyzed technical feasibility
\end{itemize}

\textbf{Conclusion: Not viable.} Image quality was insufficient for reliable detection, geographic data integration was complex, and the user experience was unconvincing.

\section{The Pivot Decision}

We decided to pivot toward a more achievable concept: a \textbf{cultural quiz on monuments and artworks} within a virtual museum environment.

\subsection{Advantages of New Concept}

\begin{enumerate}
    \item \textbf{Full control:} Controlled 3D environment, no camera dependency
    \item \textbf{AI value preserved:} Dynamic quiz generation via Gemini
    \item \textbf{Rich experience:} Immersive museum exploration
    \item \textbf{Feasibility:} Realistic with available resources
\end{enumerate}

\section{Individual/Subgroup Work}

\subsection{New Scope Definition}

\begin{itemize}
    \item Game rules: score system, timer, victory conditions
    \item Quiz format specification (question + 4 answers)
    \item Technical architecture draft
\end{itemize}

\section{Problems \& Solutions}

\begin{tabularx}{\textwidth}{|l|X|X|}
\hline
\textbf{Problem} & \textbf{Impact} & \textbf{Solution} \\
\hline
Pivot acceptance & Psychological difficulty abandoning initial work & Focus on final goal: deliver working product \\
\hline
\end{tabularx}

\section{Week Outcome}

\textbf{Abandoned:} Street View concept\\
\textbf{Defined:} Museum Quiz VR concept\\
\textbf{Decision:} Full development starting Week 5

\chapter{Week 5 -- Quiz Prototype \& UI/UX}

\section{Group Session Work}

With the new concept validated, development accelerated with parallel workstreams.

\subsection{Quiz Prototype}

\begin{itemize}
    \item Basic quiz interface created
    \item Question/answer logic implemented
    \item Gemini API integration tested
\end{itemize}

\subsection{Game Rules Definition}

\begin{table}[H]
\centering
\begin{tabular}{|l|l|}
\hline
\textbf{Parameter} & \textbf{Default Value} \\
\hline
Target score & 500 points \\
Paintings to complete & 5 paintings \\
Game duration & 5 minutes \\
Points per correct answer & 100 points \\
\hline
\end{tabular}
\caption{Game rules specification}
\end{table}

\section{Individual/Subgroup Work}

\textbf{UI/UX Subgroup (Lisa, Nina):}
\begin{itemize}
    \item VR UI guidelines definition
    \item Futuristic cyan visual style selection
    \item Screen mockups (HUD, Quiz, GameOver)
\end{itemize}

\textbf{Game Loop Subgroup (Gabriel B., Raphael):}
\begin{itemize}
    \item Flow diagram finalization
    \item Game state definitions
    \item Transition specifications
\end{itemize}

\textbf{AI Subgroup (Louis L., Louis G., Gabriel E.):}
\begin{itemize}
    \item Gemini prompt optimization
    \item JSON quiz format structuring
    \item Response reliability testing
\end{itemize}

\section{Problems \& Solutions}

\begin{tabularx}{\textwidth}{|l|X|X|}
\hline
\textbf{Problem} & \textbf{Impact} & \textbf{Solution} \\
\hline
Inconsistent AI responses & Malformed JSON occasionally & Stricter prompt with examples \\
\hline
API latency & Noticeable loading time & Loading indicator, pre-fetching \\
\hline
\end{tabularx}

\section{Week Outcome}

\textbf{Achieved:} Functional quiz prototype (non-VR), Gemini integration validated, UI guidelines defined\\
\textbf{Decision:} Full Unity/VR integration in Week 6

\chapter{Week 6 -- Unity Integration \& Playable Loop}

\section{Group Session Work}

This week marked the integration of all components in Unity. The team worked synchronously to assemble different modules.

\subsection{Museum Scene}

\begin{itemize}
    \item 3D museum assets imported
    \item 31 monument paintings placed
    \item Lighting configured
\end{itemize}

\subsection{Interaction System}

\begin{itemize}
    \item VR raycast for painting pointing
    \item UI button click detection
    \item Visual hover feedback
\end{itemize}

\section{Individual/Subgroup Work}

\subsection{Software Architecture}

The final architecture follows Unity best practices:

\textbf{Core Scripts:}
\begin{itemize}
    \item \texttt{GameManager.cs}: Singleton managing game states (MainMenu, Playing, Paused, GameOver), score, timer, and win conditions
    \item \texttt{APIManager.cs}: Handles Gemini API calls with retry logic for 503/429 errors
\end{itemize}

\textbf{Gameplay Scripts:}
\begin{itemize}
    \item \texttt{PaintingController.cs}: Stores painting data (title, artist, year, context)
    \item \texttt{PlayerInteraction.cs}: VR raycast and interaction handling
\end{itemize}

\textbf{UI Scripts:}
\begin{itemize}
    \item \texttt{HUDController.cs}: Wrist-mounted timer/score display
    \item \texttt{QuizUIController.cs}: Quiz panel with 4 answer buttons
    \item \texttt{GameOverUIController.cs}: End screen with retry/quit options
\end{itemize}

\subsection{Quiz Data Format}

\begin{verbatim}
{
  "question": "In what year was the Eiffel Tower built?",
  "trueAnswer": "1889",
  "falseAnswers": ["1900", "1875", "1920"],
  "historicalFact": "The tower was meant to be temporary..."
}
\end{verbatim}

\section{Problems \& Solutions}

\begin{tabularx}{\textwidth}{|l|X|X|}
\hline
\textbf{Problem} & \textbf{Impact} & \textbf{Solution} \\
\hline
API 503 errors & Quiz generation failures & Implemented retry system (up to 3 attempts) \\
\hline
VR Canvas setup & UI invisible or wrong scale & World Space mode, scale 0.001 \\
\hline
Button raycast & Buttons not clickable & Added BoxCollider components \\
\hline
\end{tabularx}

\section{Week Outcome}

\textbf{Achieved:} Complete museum scene, functional game loop, AI-generated quizzes, wrist HUD\\
\textbf{Decision:} Polish and stabilization in Week 7

\chapter{Week 7 -- Finalization \& Delivery}

\section{Group Session Work}

The final week focused on polish, bug fixes, and deliverable preparation.

\subsection{UI/UX Improvements}

\begin{itemize}
    \item Historical fact display after each answer
    \item Monument list on game over screen
    \item +/- buttons for menu sliders (more VR-friendly)
    \item Panel distance adjustments
\end{itemize}

\subsection{Bug Fixes}

\begin{itemize}
    \item Fixed auto-click on button hover
    \item Fixed laser visibility issues
    \item Fixed quit button (now returns to menu instead of closing app)
    \item Fixed HUD visibility in menu state
\end{itemize}

\section{Individual/Subgroup Work}

\subsection{Final Testing}

\begin{itemize}
    \item Tests on actual Meta Quest 3S hardware
    \item All interactions verified
    \item Edge cases tested (API timeout, wrong answers)
\end{itemize}

\subsection{Deliverable Preparation}

\begin{itemize}
    \item APK build for Quest
    \item Technical documentation (CLAUDE.md)
    \item Report and presentation preparation
\end{itemize}

\section{Problems \& Solutions}

\begin{tabularx}{\textwidth}{|l|X|X|}
\hline
\textbf{Problem} & \textbf{Impact} & \textbf{Solution} \\
\hline
Menu too close & Serialized scene value overriding code & Modified value directly in Unity editor \\
\hline
Text overlap in GameOver & Poor readability & Adjusted RectTransform anchors \\
\hline
\end{tabularx}

\section{Week Outcome}

\textbf{Achieved:} Stable, playable application; all critical bugs fixed; Quest build functional

% Figure: Final Interface
\begin{figure}[H]
\centering
\includegraphics[width=0.85\textwidth]{images/museum_final.png}
\caption{Final museum interface showing HUD, paintings, and VR controllers}
\label{fig:final}
\end{figure}

% Figure: Final Kanban
\begin{figure}[H]
\centering
\includegraphics[width=0.95\textwidth]{images/kanban_final.png}
\caption{Final Kanban board showing completed tasks}
\label{fig:kanban_final}
\end{figure}

\chapter{Technical Architecture}

\section{System Overview}

The application follows an event-driven architecture with singleton managers:

\begin{figure}[H]
\centering
\begin{tabular}{|c|}
\hline
\textbf{GameManager (Singleton)} \\
States: MainMenu $\rightarrow$ Playing $\leftrightarrow$ Paused $\rightarrow$ GameOver \\
\hline
\end{tabular}

\vspace{0.3cm}
$\downarrow$ Events $\downarrow$

\begin{tabular}{|c|c|c|}
\hline
\textbf{APIManager} & \textbf{UI Controllers} & \textbf{Gameplay} \\
Gemini API calls & HUD, Quiz, Menu & Paintings, Interaction \\
\hline
\end{tabular}
\caption{High-level architecture diagram}
\end{figure}

\section{Key Components}

\subsection{GameManager}

Core responsibilities:
\begin{itemize}
    \item Game state management with C\# events for state changes
    \item Score and timer tracking with update events
    \item Win/lose condition evaluation
    \item UI prefab instantiation (Quiz, GameOver, MainMenu panels)
\end{itemize}

\subsection{APIManager}

Features implemented:
\begin{itemize}
    \item HTTP POST requests to Gemini API
    \item Automatic retry on errors 503, 429, 500, 502
    \item Configurable timeout (default: 30 seconds)
    \item JSON parsing with error handling
\end{itemize}

\subsection{VR UI Configuration}

Critical settings for VR Canvas:
\begin{itemize}
    \item \textbf{Render Mode:} World Space
    \item \textbf{Scale:} 0.0005 to 0.001 (very small for VR)
    \item \textbf{Buttons:} Must have BoxCollider for raycast detection
    \item \textbf{Style:} Cyan (\#00E5FF) on dark semi-transparent background
\end{itemize}

\section{Monument Content}

31 monuments with complete data:
\begin{multicols}{2}
\begin{enumerate}[noitemsep]
    \item Eiffel Tower
    \item Colosseum
    \item Sagrada Familia
    \item Big Ben
    \item Statue of Liberty
    \item Pyramids of Giza
    \item Great Wall of China
    \item Machu Picchu
    \item Christ the Redeemer
    \item Chichen Itza
    \item Sydney Opera House
    \item Petra
    \item Hagia Sophia
    \item Palace of Versailles
    \item The Louvre
    \item Notre-Dame
    \item Mont-Saint-Michel
    \item Sacre-Coeur
    \item Arc de Triomphe
    \item Champs-Elysees
    \item Pont du Gard
    \item Nimes Arena
    \item Strasbourg Cathedral
    \item Chambord Castle
    \item Chenonceau Castle
    \item Carcassonne
    \item Papal Palace
    \item Sainte-Chapelle
    \item Opera Garnier
    \item Pantheon (Paris)
    \item US Capitol
\end{enumerate}
\end{multicols}

\chapter{Conclusion}

\section{Project Assessment}

Museum Quiz VR represents the successful completion of a seven-week iterative development process. Despite a mid-project pivot, the team delivered a functional and educational VR application.

\subsection{Goals Achieved}

\begin{itemize}
    \item[$\checkmark$] Functional VR application on Meta Quest 3S
    \item[$\checkmark$] Meaningful AI integration (Google Gemini)
    \item[$\checkmark$] Engaging user experience with 31 monuments
    \item[$\checkmark$] Complete game loop with scoring and progression
\end{itemize}

\subsection{Strengths}

\begin{enumerate}
    \item \textbf{Adaptability:} The Week 4 pivot demonstrated team maturity in prioritizing deliverables over sunk costs
    \item \textbf{AI Value:} Dynamic quiz generation provides genuine educational value and replayability
    \item \textbf{UI Quality:} The interface is both aesthetic and ergonomic for VR
\end{enumerate}

\section{Challenges Encountered}

\begin{enumerate}
    \item \textbf{Limited hardware access:} Single VR headset for 7 team members created a testing bottleneck
    \item \textbf{Unity learning curve:} Several team members were new to Unity
    \item \textbf{Meta SDK complexity:} Documentation gaps caused significant delays
    \item \textbf{API instability:} Frequent 503 errors required robust retry mechanisms
\end{enumerate}

\section{Lessons Learned}

\begin{itemize}
    \item \textbf{Validate technical feasibility early:} The initial concept's hardware requirements should have been verified before significant investment
    \item \textbf{Pivots are valuable:} Abandoning a non-viable approach is better than delivering a broken product
    \item \textbf{VR has unique constraints:} World Space Canvas, small scales, and colliders for raycast are non-obvious requirements
    \item \textbf{API error handling is critical:} External dependencies require robust fallback mechanisms
\end{itemize}

\section{What Could Have Been Better}

\begin{itemize}
    \item More VR headsets for parallel testing
    \item Smaller initial scope with clearer technical requirements
    \item Earlier prototype on actual hardware
    \item More time for performance optimization
\end{itemize}

\section{Future Improvements}

\begin{enumerate}
    \item \textbf{Leaderboard:} High score tracking and ranking
    \item \textbf{Audio:} Ambient music and sound effects
    \item \textbf{More content:} Additional monuments and quiz categories
    \item \textbf{Save system:} Progress persistence across sessions
    \item \textbf{Multiplayer:} Competitive or cooperative modes
    \item \textbf{Difficulty levels:} Adjustable quiz complexity
\end{enumerate}

\vspace{0.5cm}
\begin{center}
\textit{Museum Quiz VR demonstrates that with adaptability and focus on deliverables, constraints can be transformed into opportunities for creative problem-solving.}
\end{center}

\end{document}
